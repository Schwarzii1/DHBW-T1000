\section{IT-Asset-Inventory}

  „Ein IT-Asset [...] ist jede Art von Informationen oder Daten, Software oder Hardware, die ein Unternehmen im Rahmen seiner Geschäftstätigkeit nutzt.'' (\cite{IBM})

IT-Asset-Management (ITAM) bezieht sich auf die vollständige Nachverfolgung und Verwaltung von IT-Assets. Es stellt sicher, dass jedes IT-Asset effizient genutzt, ordnungsgemäß gewartet, regelmäßig aktualisiert und am Ende seiner Lebensdauer angemessen entsorgt wird (vgl. \cite{IBM}). Das IT-Asset-Management ist niemals ein zeitlich terminiertes Projekt, sondern ein Prozess, der regelmäßig durchgeführt werden muss, um einen aktuellen Stand zu wahren. In diesem Prozess unterteilt man die folgenden Schritte: Zu Beginn wird ein detailliertes Inventar aller IT-Assets erstellt. Im Inventar werden die gekauften IT-Assets nach Kriterien aufgelistet, welche variabel festgelegt werden können. Besonders im Bereich der OT sind sinnvolle Kriterien beispielsweise der Name des IT-Assets, der Gerätetyp, die IP-Adresse, der Standort, die Seriennummer oder auch Hersteller und Wartungsstatus des Geräts. Grundsätzlich können diese je nach Anwendung variabel gesetzt werden. Im Bereich der industriellen Cybersicherheit ist ein detailliertes OT-Asset-Inventar von entscheidender Bedeutung. Eine adäquate Sicherung von OT-Geräten und zugehöriger Software ist nur möglich, wenn diese umfassend erfasst und dokumentiert sind (\cite{atlassian}). Eine strukturierte Liste aller Geräte im Automationsnetz ermöglicht die Entwicklung effektiver Schutzkonzepte. Ein umfassendes Inventar sollte zusätzliche Informationen wie Softwarestand, Version und Betriebssystem enthalten, um grundlegende Schwachstellen schnell diagnostizieren zu können. Ein weiterer Vorteil liegt in den Kosteneinsparungen, die durch ein sorgfältig geführtes Inventar erzielt werden. Dies trägt zur Vermeidung unnötiger Anschaffungen bei und optimiert die Nutzung vorhandener Ressourcen. Zudem ermöglicht die gewonnene Transparenz eine bessere Planung und Durchführung von Wartungsarbeiten (vgl. \cite{sichereIndustrie}). Im Folgenden ist dies von besonderer Bedeutung für diese Projektarbeit, da im weiteren Verlauf eine Technologie präsentiert wird, welche die Verbesserung dieses Prozesses ermöglicht. Auf das IT-Asset-Inventar folgt nun die Nachverfolgung von IT-Assets, wobei ein ITAM-Tool verwendet wird, dass zur kontinuierlichen Überwachung dient. Daraufhin folgt die Anlagenwartung, welche "[...] die Reparatur, die Aktualisierung und den Austausch von Assets umfasst" (\cite{IBM}).