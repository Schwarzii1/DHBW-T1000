\section{Einleitung}



In einer Welt, in der industrielle Produktionsprozesse zunehmend von digitalen Technologien abhängen, wird die Sicherheit von Betriebstechnologien zu einer der größten Herausforderungen für die Industrie. Die rasante Digitalisierung hat nicht nur Effizienz und Produktivität gesteigert, sondern auch eine neue Dimension von Bedrohungen mit sich gebracht. Angriffe auf Produktionsanlagen sind keine hypothetischen Szenarien mehr, sondern reale und ernsthafte Gefahren, die erhebliche Schäden verursachen können. Ein aktuelles Beispiel mit beträchtlichem Schadensausmaß ist der Angriff auf Varta im Jahr 2024. Der Batteriehersteller meldete in der Nacht des 12. Februars einen signifikanten Sicherheitsvorfall, woraufhin die Produktion vorübergehend zum Stillstand kam. Es ist wichtig zu betonen, dass sich das Schadensausmaß solcher Vorfälle nicht nur auf das betroffene Unternehmen konzentriert. Aufgrund seiner vielseitigen Beziehungen zu verschiedenen Industriezweigen sind zwangsläufig auch Bereiche wie die Automobilindustrie betroffen (vgl. \cite{VartaAngriff}). Ein ähnliches Szenario ereignete sich bei Honda im Jahr 2020, als Anlagen in Nordamerika, Großbritannien, Italien und der Türkei zeitweise stillgelegt werden mussten (vgl. \cite{HondaAngriff}). 

Da sich gezeigt hat, dass traditionelle IT-Sicherheitsmaßnahmen nicht alle Bereiche eines produzierenden Betriebs schützen können, rückt die Sicherheit von Betriebstechnologien (OT-Security) immer mehr ins Zentrum. Es gilt nun, passende Strategien und Schutzkonzepte zu entwickeln, um diese Vorfälle zukünftig zu vermeiden.

In der OT-Security gilt der Grundsatz, dass nur bekannte IT-Objekte adäquat geschützt werden können. Daher ist ein umfassendes und aktuelles Inventory aller Assets und deren Attribute in Produktionsumgebungen essentiell, um Sicherheitsrisiken zu minimieren und Produktionsabläufe zu optimieren. Angesichts der zunehmenden Komplexität der Fertigungsprozesse und der wachsenden Anzahl vernetzter Geräte besteht die Aufgabe, ein strukturiertes und effizientes Asset Management zu etablieren. Eine präzise Bestimmung von Schutzobjekten gilt als Fundament eines jeden Sicherheitskonzepts, sowohl in der IT als auch in der OT.

Das Ziel dieser Arbeit ist es, einen theoretischen Einblick in die OT-Security zu geben und Methoden des Asset Inventorings in der Produktion der Porsche AG zu erläutern. Weiterhin wird das Thema passiver Netzwerksniffer zur Erfassung von OT-Asset-Attributen beleuchtet. In einem praktischen Feldversuch werden die Einsatzmöglichkeiten eines passiven Netzwerksniffers zur Erfassung von OT-Geräten und deren Attributen im PROFINET des Karosseriebaus der Porsche AG untersucht und bewertet. Durch einen solchen praxisorientierten Versuch in einer Automatisierungstestzelle soll gezeigt werden, wie mittels passivem und aktivem Sniffing wertvolle Daten gesammelt werden können, ohne die Verfügbarkeit und Stabilität des Produktionsbetriebs zu beeinträchtigen. Die gewonnenen Erkenntnisse sollen zur Verbesserung des Asset Inventories und zur Erhöhung der Sicherheit in der Produktion beitragen.

