\section{Einleitung}

In einer Welt, in der industrielle Produktionsprozesse zunehmend von digitalen Technologien abhängen, wird die Sicherheit von Betriebstechnologien (OT) zu einer der größten Herausforderungen für die Industrie. Die rasante Digitalisierung hat nicht nur Effizienz und Produktivität gesteigert, sondern auch eine neue Dimension von Bedrohungen mit sich gebracht. Angriffe auf Produktionsanlagen sind keine hypothetischen Szenarien mehr, sondern reale und ernsthafte Gefahren, die erhebliche Schäden verursachen können. Ein aktuelles Beispiel mit erheblichen Schadensausmaß, ist der Angriff auf Varta  2024. Der Batteriehersteller meldete in der Nacht des 12. Februar  einen signifikanten Sicherheitsvorfall, woraufhin die Produktion vorrübergehend zum Stillstand kam. Es ist stets zu beachten, dass sich das Schadensausmaß in solchen Fällen nicht nur auf Varta selbst konzentriert. Da das Unternehmen vielseitige Beziehungen in jegliche Industriezweige pflegt, sind zwangsläufig auch Bereiche der Automobilindustrie betroffen. Da sich immer wieder gezeigt hat, dass die IT-Security nicht alle Bereiche eines produzierenden Betriebes schützen kann, rückt die OT-Security für die Industrie immer mehr ins Zentrum. Es gilt nun, diese Vorfälle zukünftig zu vermeiden.

In der OT-Security gilt der Grundsatz, dass nur bekannte IT-Objekte adäquat geschützt werden können. Daher ist ein umfassendes und aktuelles Inventory aller Assets und deren Attribute in Produktionsumgebungen  essenziell, um Sicherheitsrisiken zu minimieren und Produktionsabläufe zu optimieren. Angesichts der zunehmenden Komplexität der Fertigungsprozesse und der wachsenden Anzahl vernetzter Geräte besteht die Aufgabe, ein strukturiertes und effizientes Asset Management zu etablieren. Eine präzise Bestimmung von Schutzobjekten gilt als Fundament eines jeden Sicherheitskonzeptes, sowohl in der IT als auch in der OT.

Ziel dieser Arbeit ist es, in einem Theorieteil einen Einblick in die OT-Security zu geben und Methoden des Asset Inventorings in der Produktion der Porsche AG darzulegen. Darüber hinaus wird das Thema passiver Netzwerksniffer zur Erfassung von OT-Asset-Attributen beleuchtet. In einem praktischen Feldversuch, werden die Einsatzmöglichkeiten eines passiven Netzwerk-Sniffers zur Erfassung von OT-Geräten und deren Attributen im PROFINET-Netzwerk in der Produktion im Karosseriebau zu untersuchen und zu bewerten. Durch einen praktischen Versuch soll gezeigt werden, wie mittels passivem Sniffing wertvolle Daten gesammelt werden können, ohne die Verfügbarkeit und Stabilität des Produktionsbetriebs zu beeinträchtigen. Die gewonnenen Erkenntnisse sollen zur Verbesserung des Asset Managements und zur Erhöhung der Sicherheit in der Produktion beitragen.
Diese Arbeit bietet nicht nur einen Einblick in die spezifischen Anforderungen und Herausforderungen des PROFINET-Standards, sondern auch eine praxisorientierte Analyse der Effektivität passiver Sniffing-Techniken im industriellen Umfeld. Sie trägt somit zur Schaffung eines fundierten und robusten Asset Managements bei, welches als Basis für die Sicherheitsstrategie in einer hochkomplexen und sensiblen Produktionsumgebung dient.