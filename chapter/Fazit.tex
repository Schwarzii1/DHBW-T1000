\section{Fazit}

In dieser Arbeit wurden die Analyse und Bewertung von Asset Inventory Tools zur Erfassung von OT-Geräte-Attributen im Produktionsumfeld der Porsche AG durchgeführt. Der theoretische Teil der Arbeit beleuchtet die Hintergründe der OT-Security, die sich durch spezifische Herausforderungen und Risiken deutlich von der IT-Security unterscheidet. Während in der IT-Security oftmals der Schutz von Daten im Vordergrund steht, fokussiert sich die OT-Security verstärkt auf die Integrität und Verfügbarkeit von industriellen Steuerungssystemen und Produktionsumgebungen. Diese Differenzierung ist notwendig, um den unterschiedlichen Prioritäten der Schutzziele gerecht zu werden. Die Relevanz dieses Themas ist in den letzten Jahren stark gestiegen, da die Angriffsszenarien in einer immer höheren Geschwindigkeit komplexer und dynamischer werden. Unternehmen und staatliche Institutionen mussten auf zunehmend erhebliche Angriffe reagieren, was die dringende Notwendigkeit unterstreicht, sowohl präventive als auch reaktive Sicherheitsmaßnahmen zu implementieren. Besonders KRITIS sind hierbei von Bedeutung, da Angriffe auf diese Systeme erhebliche Auswirkungen auf die Zivilbevölkerung haben können. Im praktischen Teil der Arbeit wurden zwei Asset Inventory Tools – das DPM-System und der PROFINET Inspektor von Indu-Sol – hinsichtlich ihrer Leistungsfähigkeit bei der Erfassung von OT-Geräten analysiert. Beide Systeme weisen hohe Erkennungsgrade auf. DPM und Indu-Sol haben grundsätzlich ähnliche Erkennungsgrade, jedoch liegen signifikante Diskrepanzen in der Datenqualität vor. Insbesondere im Attribut Gerätetyp liefert der PROFINET Inspektor ungenaue Daten. Dies ist besonders für das Patch-Management und die Bewertung von Sicherheitsrisiken relevant, da viele Sicherheitsupdates nur für spezifische Modelle verfügbar sind. Diese genaue Erfassung von Gerätetypen ist für die OT-Security von großer Bedeutung, da sie sicherstellt, dass die richtigen Updates und Sicherheitsmaßnahmen angewendet werden können. Die Kombination der beiden Systeme könnte eine umfassendere Netzwerksicht bieten, da sie jeweils unterschiedliche Protokolle und Geräte erkennen können. Ein weiteres Unterscheidungsmerkmal ist die Protokollunterstützung: DPM kann neben PROFINET auch andere Protokolle wie Modbus, OPC UA und Ethernet erkennen, während der PROFINET Inspektor nur PROFINET unterstützt. Dadurch bietet DPM eine vielseitigere Lösung für das Asset Inventory. Insbesondere die Nutzung der GSDML-Dateien seitens DPM liefert hierbei einen entscheidenden Vorteil. Es bestehen jedoch auch Sicherheitsrisiken bei beiden Tools. Die Zusammenarbeit des PROFINET Inspektors mit externen Firmen birgt potenzielle Zugriffsrisiken. 
Letztlich lässt sich feststellen, dass DPM aufgrund der breiteren Protokollunterstützung und besseren Datenqualität eine vielseitigere Lösung darstellt. Eine kombinierte Nutzung beider Tools könnte jedoch eine noch umfassendere Netzwerksicht und erhöhte Sicherheit gewährleisten.