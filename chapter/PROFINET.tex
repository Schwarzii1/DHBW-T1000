\section{PROFINET}

PROFINET ist ein Industrial-Ethernet-Standard, der speziell für Anwendungen in der industriellen Automatisierung, einschließlich der Fertigungs- und Prozessautomatisierung, entwickelt wurde. Er ermöglicht die deterministische Datenkommunikation\footnote{bevor auf das Netzwerk zugegriffen werden kann, muss eine Rechtevergabe erfolgen, bei der nur ein Host das Senderecht und die Nutzung eines Kanals zugesichert bekommt} und zeichnet sich durch seine Modularität und Unterstützung verschiedener Übertragungsmedien aus, wobei er auf TCP/IP basiert. Der Standard definiert vier Konformitätsklassen, um unterschiedlichen Funktionen und Anforderungen an die Echtzeitkommunikation gerecht zu werden (vgl. \cite{IPInsider}). PROFINET definiert vier aufeinander aufbauende Konformitätsklassen (Conformance Classes, CC). Diese Klassen sind jeweils spezifisch auf den entsprechenden Einsatzzweck abgestimmt:

\begin{itemize}
\item \textbf{CC-A} enthält Grundfunktionen.
\item \textbf{CC-B} erweitert CC-A um Netzwerkdiagnose und Topologie-Informationen.
\item \textbf{CC-C} umfasst eine Erweiterung zur Implementierung von IRT-Kommunikation\footnote{(Isochronous Real-Time Communication) synchronisiertes Übertragungsverfahren, das für den zyklischen Austausch von Daten zwischen PROFINET-Geräten verwendet wird}, die die Grundlage für taktsynchrone Anwendungen bildet.
\item \textbf{CC-D} erweitert die Klasse C, indem es dieselben Dienste unter Verwendung der von IEEE (Institute of Electrical and Electronics Engineers) definierten Mechanismen des TSN (Time-Sensitive Networking) bereitstellt.
\end{itemize}

PROFINET nutzt ein einfaches 4-schichtiges TCP/IP-Modell.  










