%%%%%%%%%%%%%%%%%%%%%%%%%%%%%%%%%%%%%%%%%%%%%%%%%%%%%%%%%%%%%%%%%%%%
%%%           Vorlage für eine Ausarbeitung an der DHBW          %%%
%%%                                                              %%%
%%%      Bereiche die bearbeitet werden müssen werden durch      %%%
%%%      einen solchen Kommentarblock eingeleitet und enden      %%%
%%%      mit der nächsten Trennlinie.                            %%%
%%%                                                              %%%
%%%      In dieser Datei müssen folgende Bereiche bearbeitet     %%%
%%%      werden:                                                 %%%
%%%      - Angaben zur Arbeit                                    %%%
%%%      - EIGENE KAPITEL EINFÜGEN                               %%%
%%%                                                              %%%
%%%      Benötigte Seiten und Verzeichnisse können unter         %%%
%%%      "Einführung und Verzeichnisse" ein- bzw. auskommentiert %%%
%%%      werden.                                                 %%%
%%%                                                              %%%
%%%%%%%%%%%%%%%%%%%%%%%%%%%%%%%%%%%%%%%%%%%%%%%%%%%%%%%%%%%%%%%%%%%%

\documentclass[a4paper,12pt]{article}
\usepackage[left=2.5cm,right=2.5cm,top=2.5cm,bottom=2.5cm,includehead]{geometry}      % Einstellungen der Seitenränder
\usepackage[english, ngerman]{babel}                                                  % deutsche Silbentrennung
\usepackage[utf8]{inputenc}                                                           % Umlaute
\usepackage[official]{eurosym}                                                        % Euro Symbol
\usepackage[T1]{fontenc}													                                    % Umlaute auch richtig ausgeben
\usepackage{newtxtext,newtxmath}                                                      % Font = Times New Roman
\usepackage{hyperref}
\usepackage[nottoc]{tocbibind}
\usepackage{fancyhdr}
\usepackage{setspace}
\usepackage[backend=bibtex, citestyle=authoryear, bibstyle=authoryear]{biblatex}      % Bibliothek für Zitate
\usepackage{csquotes}                                                                 % Zusatzpacket für Zitate
\usepackage{amsmath}                                                                  % Zurücksetzen der Tabellen- und Abbildungsnummerierung je Sektion
\usepackage[labelfont=bf,aboveskip=1mm]{caption}                                      % Bild- und Tabellenunterschrift (fett)
\usepackage[bottom,multiple,hang,marginal]{footmisc}                                  % Fußnoten [Ausrichtung unten, Trennung durch Seperator bei mehreren Fußnoten]
\usepackage{graphicx}  
\graphicspath{{./images/}}                                                            % Grafiken
\usepackage[dvipsnames]{xcolor}                                                       % Farbige Buchstaben
\usepackage{wrapfig}                                                                  % Bilder in Text integrieren
\usepackage{enumitem}                                                                 % Befehl setlist (Zeilenabstand für itemize Umgebung auf 1 setzen)
\usepackage{listings}                                                                 % Quelltexte
\definecolor{commentgreen}{RGB}{87,166,74}                                            % Kommentar-Farbe für Quellcode
\lstset{numbers=left, numberstyle=\tiny, numbersep=8pt, frame=single, framexleftmargin=15pt, breaklines=true, commentstyle=\color{commentgreen}}
\usepackage{tabularx}                                                                 % Tabellen
\usepackage{multirow}                                                                 % Mehrzeilige Tabelleneinträge
\usepackage[addtotoc]{abstract}                                                       % Abstract
\usepackage[nohyperlinks, printonlyused, withpage]{acronym}                           % Abkürzungen
\usepackage{dirtree}                                                                  % Ordnerstruktur (z.B. für Anhang)

%%%%%%%%%%%%%%%%%%%%%%%%%%%%%%%%%%%%%%%%%%%%%%%%%%%%%%%%%%%%%%%%%%%%
%%%                      Angaben zur Arbeit                      %%%
%%%%%%%%%%%%%%%%%%%%%%%%%%%%%%%%%%%%%%%%%%%%%%%%%%%%%%%%%%%%%%%%%%%%
\def\vFirmenlogoPfad{images/Porsche_Wappen.png}                  %% relativer Pfad Bsp.: images/Firmenlogo.png
\def\vDHBWLogoPfad{images/DHBW_logo.jpg}                          %% relativer Pfad Bsp.: images/DHBW_logo.jpg
\def\vUnterschrift{}                    %% Pfad zu Bild mit Unterschrift (für digitale Abgabe) Bsp.: images/Unterschrift.png

\def\vTitel{Analyse und Bewertung eines passiven Netzwerksniffers zur Erfassung von OT-Geräte-Attributen im PROFINET}                           %% 
\def\vUntertitel{}                      %% 
\def\vArbeitstyp{Projektarbeit}                      %% Projektarbeit/Seminararbeit/Bachelorarbeit
\def\vArbeitsbezeichnung{T1000}              %% T1000/T2000/T3000

\def\vAutor{Janik Schwarzenberger}                           %% Vorname Nachname
\def\vMatrikelnummer{5229478}                  %% 7-stellige Zahl
\def\vKursKuerzel{TIS23}                     %% Bsp.: TIT20
\def\vPhasenbezeichnung{Praxisphase}               %% Praxisphase/Theoriephase
\def\vStudienJahr{erste}                     %% erste/zweite/dritte
\def\vDHBWStandort{Ravensburg}                    %% Bsp.: Ravensburg
\def\vDHBWCampus{Friedrichshafen}                      %% Bsp.: Friedrichshafen
\def\vFakultaet{Technik}                       %% Technik/Wirtschaft
\def\vStudiengang{Informatik IT-Security}                     %% Informationstechnik/...

\def\vBetrieb{Dr. Ing h. c. F. Porsche AG}                         %% 
\def\vBearbeitungsort{Stuttgart}                 %% 
\def\vAbteilung{}                       %% 
\def\vBetreuer{Alexander Bock}                        %% Vorname Nachname

\def\vAbgabedatum{\today}               %% DD. MONTH YYYY
\def\vBearbeitungszeitraum{01.07.2024 - 30.09.2024}            %% DD.MM.YYYY - DD.MM.YYYY


%%%%%%%%%%%%%%%%%%%%%%%%% Eigene Kommandos %%%%%%%%%%%%%%%%%%%%%%%%%
% Definition von \gqq{} und \gq{}: Text in Anführungszeichen
\newcommand{\gqq}[1]{\glqq #1\grqq}
\newcommand{\gq}[1]{\glq #1\grq}
% Spezielle Hervorhebung von Schlüsselwörtern
\newcommand{\textOrdner}[1]{\texttt{#1}}
\newcommand{\textVariable}[1]{\texttt{#1}}
\newcommand{\textKlasse}[1]{\texttt{#1}}
\newcommand{\textFunktion}[1]{\texttt{#1}}


%%%%%%%%%%%%%%%%%%%% Zitatbibliothek einbinden %%%%%%%%%%%%%%%%%%%%%
\addbibresource{./literatur/literatur.bib}


%%%%%%%%%%%%%%%%%%%%%%%% PDF-Einstellungen %%%%%%%%%%%%%%%%%%%%%%%%%
\hypersetup{
  bookmarksopen=false,
	bookmarksnumbered=true,
	bookmarksopenlevel=0,
  pdftitle=\vTitel,
  pdfsubject=\vTitel,
  pdfauthor=\vAutor,
  pdfborder={0 0 0},
	pdfstartview=Fit,
  pdfpagelayout=SinglePage
}


%%%%%%%%%%%%%%%%%%%%%%%% Kopf- und Fußzeile %%%%%%%%%%%%%%%%%%%%%%%%
\pagestyle{fancy}
\setlength{\headheight}{15pt}
\fancyhf{}
\fancyhead[R]{\thepage}


%%%%%%%%%%%%%%%%%%%%%%%%%%%%%% Layout %%%%%%%%%%%%%%%%%%%%%%%%%%%%%%
\onehalfspacing
\setlist{noitemsep}

\addto\captionsngerman{
  \renewcommand{\figurename}{Abb.}
  \renewcommand{\tablename}{Tab.}
}
\numberwithin{table}{section}                               % Tabellennummerierung je Sektion zurücksetzen
\numberwithin{figure}{section}                              % Abbildungsnummerierung je Sektion zurücksetzen
\renewcommand{\thetable}{\arabic{section}.\arabic{table}}   % Tabellennummerierung mit Section
\renewcommand{\thefigure}{\arabic{section}.\arabic{figure}} % Abbildungsnummerierung mit Section
\renewcommand{\thefootnote}{\arabic{footnote}}              % Sektionsbezeichnung von Fußnoten entfernen

\renewcommand{\multfootsep}{, }                             % Mehrere Fußnoten durch ", " trennen


%%%%%%%%%%%%%%%%%%%%%%%%%%%%% Dokument %%%%%%%%%%%%%%%%%%%%%%%%%%%%%

\begin{document}


  %%%%%%%%%%%%%%%%%%% Einführung und Verzeichnisse %%%%%%%%%%%%%%%%%%%
  \pagenumbering{Roman}

  \begin{titlepage}
  \begin{minipage}{6in}
    \vspace*{-2cm}
    \centering
    \hspace{-2cm}
	\ifx\vFirmenlogoPfad\empty
	\else
    \raisebox{-0.5\height}{\includegraphics[height=4cm]{\vFirmenlogoPfad}}
  \fi
	\hfill
	\ifx\vDHBWLogoPfad\empty
	\else
   	\raisebox{-0.5\height}{\includegraphics[height=4cm]{\vDHBWLogoPfad}}
	\fi
  \end{minipage}
  \begin{center}
    \vspace*{0.5cm}
    \Huge\textbf{\vTitel}\\
		\ifx\vUntertitel\empty
		\else
			\Large\rm\vUntertitel\\
		\fi
		\vspace*{2cm}
		\Large\textbf{\vArbeitstyp}
		\ifx\vArbeitsbezeichnung\empty
		\else
			\textbf{\vArbeitsbezeichnung}
		\fi
		\\
		\normalsize
		über die \vPhasenbezeichnung\ des \vStudienJahr{n}\ Studienjahrs \\
		\vspace*{1cm}
		an der Fakultät für \vFakultaet\\
		im Studiengang \vStudiengang\\
		\vspace*{0.5cm}
		an der DHBW \vDHBWStandort\\
		\ifx\vDHBWCampus\empty
		\else
		Campus \vDHBWCampus\\
		\fi
		\vspace*{0.5cm}
		von\\
		\ifx\vAutor\empty
		\else
			\vAutor\\
		\fi
		\vspace*{1cm}
		\vAbgabedatum
		\vfill
  \end{center}
  \begin{tabular}{ll}
    Bearbeitungszeitraum:&\vBearbeitungszeitraum\\
    Matrikelnummer, Kurs:&\vMatrikelnummer, \vKursKuerzel\\
	  Dualer Partner:&\vBetrieb\\
	  Betreuer des Dualen Partners:&\vBetreuer\\
  \end{tabular}
\end{titlepage}
\newpage
\setcounter{page}{2}
  % \thispagestyle{empty}
\section*{\Huge{Sperrvermerk}}

\addcontentsline{toc}{section}{Sperrvermerk}
gemäß Ziffer 1.1.13 der Anlage 1 zu §§ 3, 4 und 5  der Studien- und Prüfungsordnung für die Bachelorstudiengänge im Studienbereich Technik der Dualen Hochschule Baden-Würt­tem­berg vom 29.09.2017.\\

\noindent \gqq{Der Inhalt dieser Arbeit darf weder als Ganzes noch in Auszügen Personen außerhalb des Prüfungsprozesses und des Evaluationsverfahrens zugänglich gemacht werden, sofern keine anders lautende Genehmigung vom Dualen Partner vorliegt.}

\vfill
\leavevmode
\newline
\parbox{6cm}{\strut\centering \vBearbeitungsort, \vAbgabedatum\hrule\strut\centering\footnotesize Ort, Datum} 
\hfill
\ifx\vUnterschrift\empty
\parbox{6cm}{\strut\hspace{1pt} \vAbteilung\hrule\strut\centering\footnotesize Abteilung, Unterschrift}
\else
\parbox{6cm}{\strut\hspace{1pt} \vAbteilung, \parbox[b]{3cm}{\vspace{-10cm}\includegraphics[width=3cm]{\vUnterschrift}}\hrule\strut\centering\footnotesize Abteilung, Unterschrift}
\fi
\vspace{1cm}

\newpage
  \thispagestyle{empty}
\section*{\Huge{Selbstständigkeitserklärung}}

\addcontentsline{toc}{section}{Selbstständigkeitserklärung}
gemäß Ziffer 1.1.13 der Anlage 1 zu §§ 3, 4 und 5  der Studien- und Prüfungsordnung für die Bachelorstudiengänge im Studienbereich Technik der Dualen Hochschule Baden-Würt­tem­berg vom 29.09.2017.

\noindent Ich versichere hiermit, dass ich meine Bachelorarbeit (bzw. Projektarbeit oder Studienarbeit bzw. Hausarbeit) mit dem Thema: 
\begin{center}
	\Large\textbf{\vTitel}
\end{center}
selbstständig verfasst und keine anderen als die angegebenen Quellen und Hilfsmittel benutzt habe. Ich versichere zudem, dass die eingereichte elektronische Fassung mit der gedruckten Fassung übereinstimmt.

\vfill
\leavevmode
\newline
\parbox{6cm}{\strut\centering \vBearbeitungsort, \vAbgabedatum\hrule\strut\centering\footnotesize Ort, Datum} 
\hfill
\ifx\vUnterschrift\empty
\parbox{6cm}{\strut\hspace{1pt} \vAbteilung, \hrule\strut\centering\footnotesize Abteilung, Unterschrift}
\else
\parbox{6cm}{\strut\hspace{1pt} \vAbteilung, \parbox[b]{3cm}{\vspace{-10cm}\includegraphics[width=3cm]{\vUnterschrift}}\hrule\strut\centering\footnotesize Abteilung, Unterschrift}
\fi
\vspace{1cm}

\newpage
  \phantomsection
\newenvironment{keywords}{
	\begin{flushleft}
	\small	
	\textbf{
		\iflanguage{ngerman}{Schlüsselwörter}{\iflanguage{english}{Keywords}{}}
	}
}{\end{flushleft}}

% Deutsche Zusammenfassung
\begin{abstract}
	
\end{abstract}

% Schlüsselwörter Deutsch
\begin{keywords}
	
\end{keywords}


\selectlanguage{english}
% Englisches Abstract
\begin{abstract}

\end{abstract}

% Schlüsselwörter Englisch
\begin{keywords}

\end{keywords}


\selectlanguage{ngerman}
\newpage
  \tableofcontents
\newpage
  \section*{Abkürzungsverzeichnis}
\addcontentsline{toc}{section}{Abkürzungsverzeichnis}
\begin{acronym}
  \acro{DHBW}[DHBW]{Duale Hochschule Ba\-den-\-Würt\-tem\-berg}
  \acroplural{DHBW}[DHBW]{Dualen Hochschule Ba\-den-\-Würt\-tem\-berg}
\end{acronym}
\newpage
  \listoffigures
\newpage
  \listoftables
\newpage
  \lstlistoflistings
\addcontentsline{toc}{section}{Listings}
\newpage
  % \section*{Vorwort}
\addcontentsline{toc}{section}{Vorwort}
\newpage


  %%%%%%%%%%%%%%%%%%%%%%%%%%%%% Kapitel %%%%%%%%%%%%%%%%%%%%%%%%%%%%%%
  \pagestyle{fancy}
  \fancyhead[L]{\nouppercase{\rightmark}}    % Abschnittsname im Header
  \pagenumbering{arabic}

  %%%%%%%%%%%%%%%%%%%%%%%%%%%%%%%%%%%%%%%%%%%%%%%%%%%%%%%%%%%%%%%%%%%%
  %%%%                   EIGENE KAPITEL EINFÜGEN                  %%%%
  %%%%%%%%%%%%%%%%%%%%%%%%%%%%%%%%%%%%%%%%%%%%%%%%%%%%%%%%%%%%%%%%%%%%
  \section{Einleitung}



In einer Welt, in der industrielle Produktionsprozesse zunehmend von digitalen Technologien abhängen, wird die Sicherheit von Betriebstechnologien zu einer der größten Herausforderungen für die Industrie. Die rasante Digitalisierung hat nicht nur Effizienz und Produktivität gesteigert, sondern auch eine neue Dimension von Bedrohungen mit sich gebracht. Angriffe auf Produktionsanlagen sind keine hypothetischen Szenarien mehr, sondern reale und ernsthafte Gefahren, die erhebliche Schäden verursachen können. Ein aktuelles Beispiel mit beträchtlichem Schadensausmaß ist der Angriff auf Varta im Jahr 2024. Der Batteriehersteller meldete in der Nacht des 12. Februars einen signifikanten Sicherheitsvorfall, woraufhin die Produktion vorübergehend zum Stillstand kam. Es ist wichtig zu betonen, dass sich das Schadensausmaß solcher Vorfälle nicht nur auf das betroffene Unternehmen konzentriert. Aufgrund seiner vielseitigen Beziehungen zu verschiedenen Industriezweigen sind zwangsläufig auch Bereiche wie die Automobilindustrie betroffen (vgl. \cite{VartaAngriff}). Ein ähnliches Szenario ereignete sich bei Honda im Jahr 2020, als Anlagen in Nordamerika, Großbritannien, Italien und der Türkei zeitweise stillgelegt werden mussten (vgl. \cite{HondaAngriff}). 

Da sich gezeigt hat, dass traditionelle IT-Sicherheitsmaßnahmen nicht alle Bereiche eines produzierenden Betriebs schützen können, rückt die Sicherheit von Betriebstechnologien (OT-Security) immer mehr ins Zentrum. Es gilt nun, passende Strategien und Schutzkonzepte zu entwickeln, um diese Vorfälle zukünftig zu vermeiden.

In der OT-Security gilt der Grundsatz, dass nur bekannte IT-Objekte adäquat geschützt werden können. Daher ist ein umfassendes und aktuelles Inventory aller Assets und deren Attribute in Produktionsumgebungen essentiell, um Sicherheitsrisiken zu minimieren und Produktionsabläufe zu optimieren. Angesichts der zunehmenden Komplexität der Fertigungsprozesse und der wachsenden Anzahl vernetzter Geräte besteht die Aufgabe, ein strukturiertes und effizientes Asset Management zu etablieren. Eine präzise Bestimmung von Schutzobjekten gilt als Fundament eines jeden Sicherheitskonzepts, sowohl in der IT als auch in der OT.

Das Ziel dieser Arbeit ist es, einen theoretischen Einblick in die OT-Security zu geben und Methoden des Asset Inventorings in der Produktion der Porsche AG zu erläutern. Weiterhin wird das Thema passives und aktives Netzwerkscanning zur Erfassung von OT-Asset-Attributen beleuchtet. In einem praktischen Feldversuch werden die Einsatzmöglichkeiten der Asset-Inventory Tools zur Erfassung von OT-Geräten und deren Attributen im PROFINET des Karosseriebaus der Porsche AG untersucht und bewertet. Durch einen solchen praxisorientierten Versuch in einer Automatisierungstestzelle soll gezeigt werden, wie mittels Scanning wertvolle Daten gesammelt werden können, ohne die Verfügbarkeit und Stabilität des Produktionsbetriebs zu beeinträchtigen. Die gewonnenen Erkenntnisse sollen zur Verbesserung des Asset Inventories und zur Erhöhung der Sicherheit in der Produktion beitragen.


  \section{OT-Security}





  

  %%%%%%%%%%%%%%%%%%%%%%% Literaturverzeichnis %%%%%%%%%%%%%%%%%%%%%%%
  \phantomsection
\addcontentsline{toc}{section}{Literatur}
\setlength\bibitemsep{1.5\itemsep}
\printbibliography
\newpage



  %%%%%%%%%%%%%%%%%%%%%%%%%%%%%% Anhang %%%%%%%%%%%%%%%%%%%%%%%%%%%%%%
  \renewcommand{\thetable}{\Alph{section}.\arabic{table}}
  \renewcommand{\thefigure}{\Alph{section}.\arabic{figure}}
  \renewcommand{\thelstlisting}{\Alph{section}.\arabic{lstlisting}}
  \pagenumbering{Alph}

  \begin{appendix}
  \section{Anhang}
\end{appendix}
\end{document}